% Options for packages loaded elsewhere
\PassOptionsToPackage{unicode}{hyperref}
\PassOptionsToPackage{hyphens}{url}
\PassOptionsToPackage{dvipsnames,svgnames,x11names}{xcolor}
%
\documentclass[
  letterpaper,
  DIV=11,
  numbers=noendperiod]{scrartcl}

\usepackage{amsmath,amssymb}
\usepackage{iftex}
\ifPDFTeX
  \usepackage[T1]{fontenc}
  \usepackage[utf8]{inputenc}
  \usepackage{textcomp} % provide euro and other symbols
\else % if luatex or xetex
  \usepackage{unicode-math}
  \defaultfontfeatures{Scale=MatchLowercase}
  \defaultfontfeatures[\rmfamily]{Ligatures=TeX,Scale=1}
\fi
\usepackage{lmodern}
\ifPDFTeX\else  
    % xetex/luatex font selection
\fi
% Use upquote if available, for straight quotes in verbatim environments
\IfFileExists{upquote.sty}{\usepackage{upquote}}{}
\IfFileExists{microtype.sty}{% use microtype if available
  \usepackage[]{microtype}
  \UseMicrotypeSet[protrusion]{basicmath} % disable protrusion for tt fonts
}{}
\makeatletter
\@ifundefined{KOMAClassName}{% if non-KOMA class
  \IfFileExists{parskip.sty}{%
    \usepackage{parskip}
  }{% else
    \setlength{\parindent}{0pt}
    \setlength{\parskip}{6pt plus 2pt minus 1pt}}
}{% if KOMA class
  \KOMAoptions{parskip=half}}
\makeatother
\usepackage{xcolor}
\setlength{\emergencystretch}{3em} % prevent overfull lines
\setcounter{secnumdepth}{-\maxdimen} % remove section numbering
% Make \paragraph and \subparagraph free-standing
\ifx\paragraph\undefined\else
  \let\oldparagraph\paragraph
  \renewcommand{\paragraph}[1]{\oldparagraph{#1}\mbox{}}
\fi
\ifx\subparagraph\undefined\else
  \let\oldsubparagraph\subparagraph
  \renewcommand{\subparagraph}[1]{\oldsubparagraph{#1}\mbox{}}
\fi

\usepackage{color}
\usepackage{fancyvrb}
\newcommand{\VerbBar}{|}
\newcommand{\VERB}{\Verb[commandchars=\\\{\}]}
\DefineVerbatimEnvironment{Highlighting}{Verbatim}{commandchars=\\\{\}}
% Add ',fontsize=\small' for more characters per line
\usepackage{framed}
\definecolor{shadecolor}{RGB}{241,243,245}
\newenvironment{Shaded}{\begin{snugshade}}{\end{snugshade}}
\newcommand{\AlertTok}[1]{\textcolor[rgb]{0.68,0.00,0.00}{#1}}
\newcommand{\AnnotationTok}[1]{\textcolor[rgb]{0.37,0.37,0.37}{#1}}
\newcommand{\AttributeTok}[1]{\textcolor[rgb]{0.40,0.45,0.13}{#1}}
\newcommand{\BaseNTok}[1]{\textcolor[rgb]{0.68,0.00,0.00}{#1}}
\newcommand{\BuiltInTok}[1]{\textcolor[rgb]{0.00,0.23,0.31}{#1}}
\newcommand{\CharTok}[1]{\textcolor[rgb]{0.13,0.47,0.30}{#1}}
\newcommand{\CommentTok}[1]{\textcolor[rgb]{0.37,0.37,0.37}{#1}}
\newcommand{\CommentVarTok}[1]{\textcolor[rgb]{0.37,0.37,0.37}{\textit{#1}}}
\newcommand{\ConstantTok}[1]{\textcolor[rgb]{0.56,0.35,0.01}{#1}}
\newcommand{\ControlFlowTok}[1]{\textcolor[rgb]{0.00,0.23,0.31}{#1}}
\newcommand{\DataTypeTok}[1]{\textcolor[rgb]{0.68,0.00,0.00}{#1}}
\newcommand{\DecValTok}[1]{\textcolor[rgb]{0.68,0.00,0.00}{#1}}
\newcommand{\DocumentationTok}[1]{\textcolor[rgb]{0.37,0.37,0.37}{\textit{#1}}}
\newcommand{\ErrorTok}[1]{\textcolor[rgb]{0.68,0.00,0.00}{#1}}
\newcommand{\ExtensionTok}[1]{\textcolor[rgb]{0.00,0.23,0.31}{#1}}
\newcommand{\FloatTok}[1]{\textcolor[rgb]{0.68,0.00,0.00}{#1}}
\newcommand{\FunctionTok}[1]{\textcolor[rgb]{0.28,0.35,0.67}{#1}}
\newcommand{\ImportTok}[1]{\textcolor[rgb]{0.00,0.46,0.62}{#1}}
\newcommand{\InformationTok}[1]{\textcolor[rgb]{0.37,0.37,0.37}{#1}}
\newcommand{\KeywordTok}[1]{\textcolor[rgb]{0.00,0.23,0.31}{#1}}
\newcommand{\NormalTok}[1]{\textcolor[rgb]{0.00,0.23,0.31}{#1}}
\newcommand{\OperatorTok}[1]{\textcolor[rgb]{0.37,0.37,0.37}{#1}}
\newcommand{\OtherTok}[1]{\textcolor[rgb]{0.00,0.23,0.31}{#1}}
\newcommand{\PreprocessorTok}[1]{\textcolor[rgb]{0.68,0.00,0.00}{#1}}
\newcommand{\RegionMarkerTok}[1]{\textcolor[rgb]{0.00,0.23,0.31}{#1}}
\newcommand{\SpecialCharTok}[1]{\textcolor[rgb]{0.37,0.37,0.37}{#1}}
\newcommand{\SpecialStringTok}[1]{\textcolor[rgb]{0.13,0.47,0.30}{#1}}
\newcommand{\StringTok}[1]{\textcolor[rgb]{0.13,0.47,0.30}{#1}}
\newcommand{\VariableTok}[1]{\textcolor[rgb]{0.07,0.07,0.07}{#1}}
\newcommand{\VerbatimStringTok}[1]{\textcolor[rgb]{0.13,0.47,0.30}{#1}}
\newcommand{\WarningTok}[1]{\textcolor[rgb]{0.37,0.37,0.37}{\textit{#1}}}

\providecommand{\tightlist}{%
  \setlength{\itemsep}{0pt}\setlength{\parskip}{0pt}}\usepackage{longtable,booktabs,array}
\usepackage{calc} % for calculating minipage widths
% Correct order of tables after \paragraph or \subparagraph
\usepackage{etoolbox}
\makeatletter
\patchcmd\longtable{\par}{\if@noskipsec\mbox{}\fi\par}{}{}
\makeatother
% Allow footnotes in longtable head/foot
\IfFileExists{footnotehyper.sty}{\usepackage{footnotehyper}}{\usepackage{footnote}}
\makesavenoteenv{longtable}
\usepackage{graphicx}
\makeatletter
\def\maxwidth{\ifdim\Gin@nat@width>\linewidth\linewidth\else\Gin@nat@width\fi}
\def\maxheight{\ifdim\Gin@nat@height>\textheight\textheight\else\Gin@nat@height\fi}
\makeatother
% Scale images if necessary, so that they will not overflow the page
% margins by default, and it is still possible to overwrite the defaults
% using explicit options in \includegraphics[width, height, ...]{}
\setkeys{Gin}{width=\maxwidth,height=\maxheight,keepaspectratio}
% Set default figure placement to htbp
\makeatletter
\def\fps@figure{htbp}
\makeatother

\KOMAoption{captions}{tableheading}
\makeatletter
\makeatother
\makeatletter
\makeatother
\makeatletter
\@ifpackageloaded{caption}{}{\usepackage{caption}}
\AtBeginDocument{%
\ifdefined\contentsname
  \renewcommand*\contentsname{Table of contents}
\else
  \newcommand\contentsname{Table of contents}
\fi
\ifdefined\listfigurename
  \renewcommand*\listfigurename{List of Figures}
\else
  \newcommand\listfigurename{List of Figures}
\fi
\ifdefined\listtablename
  \renewcommand*\listtablename{List of Tables}
\else
  \newcommand\listtablename{List of Tables}
\fi
\ifdefined\figurename
  \renewcommand*\figurename{Figure}
\else
  \newcommand\figurename{Figure}
\fi
\ifdefined\tablename
  \renewcommand*\tablename{Table}
\else
  \newcommand\tablename{Table}
\fi
}
\@ifpackageloaded{float}{}{\usepackage{float}}
\floatstyle{ruled}
\@ifundefined{c@chapter}{\newfloat{codelisting}{h}{lop}}{\newfloat{codelisting}{h}{lop}[chapter]}
\floatname{codelisting}{Listing}
\newcommand*\listoflistings{\listof{codelisting}{List of Listings}}
\makeatother
\makeatletter
\@ifpackageloaded{caption}{}{\usepackage{caption}}
\@ifpackageloaded{subcaption}{}{\usepackage{subcaption}}
\makeatother
\makeatletter
\@ifpackageloaded{tcolorbox}{}{\usepackage[skins,breakable]{tcolorbox}}
\makeatother
\makeatletter
\@ifundefined{shadecolor}{\definecolor{shadecolor}{rgb}{.97, .97, .97}}
\makeatother
\makeatletter
\makeatother
\makeatletter
\makeatother
\ifLuaTeX
  \usepackage{selnolig}  % disable illegal ligatures
\fi
\IfFileExists{bookmark.sty}{\usepackage{bookmark}}{\usepackage{hyperref}}
\IfFileExists{xurl.sty}{\usepackage{xurl}}{} % add URL line breaks if available
\urlstyle{same} % disable monospaced font for URLs
\hypersetup{
  colorlinks=true,
  linkcolor={blue},
  filecolor={Maroon},
  citecolor={Blue},
  urlcolor={Blue},
  pdfcreator={LaTeX via pandoc}}

\author{}
\date{}

\begin{document}
\ifdefined\Shaded\renewenvironment{Shaded}{\begin{tcolorbox}[frame hidden, enhanced, boxrule=0pt, interior hidden, sharp corners, breakable, borderline west={3pt}{0pt}{shadecolor}]}{\end{tcolorbox}}\fi

\begin{Shaded}
\begin{Highlighting}[]
\CommentTok{\# Sim A}
\end{Highlighting}
\end{Shaded}

\begin{Shaded}
\begin{Highlighting}[]
\ImportTok{import}\NormalTok{ sys}
\ImportTok{from}\NormalTok{ pathlib }\ImportTok{import}\NormalTok{ Path}
\ImportTok{import}\NormalTok{ numpy }\ImportTok{as}\NormalTok{ np}
\ImportTok{import}\NormalTok{ torch}
\ImportTok{from}\NormalTok{ torch.nn.utils }\ImportTok{import}\NormalTok{ clip\_grad\_value\_}
\ImportTok{from}\NormalTok{ torch }\ImportTok{import}\NormalTok{ nn}
\ImportTok{from}\NormalTok{ sklearn.linear\_model }\ImportTok{import}\NormalTok{ PoissonRegressor}
\ImportTok{import}\NormalTok{ matplotlib.pyplot }\ImportTok{as}\NormalTok{ plt}

\NormalTok{sys.path.append(Path(}\StringTok{"low\_rank\_MPLN"}\NormalTok{))}
\ImportTok{import}\NormalTok{ low\_rank\_MPLN}

\KeywordTok{def}\NormalTok{ gen\_Sigma(num\_features, rho}\OperatorTok{=}\FloatTok{0.8}\NormalTok{):}
    \CommentTok{\# AR1 parameters}

    \CommentTok{\# Generate AR1 covariance matrix}
\NormalTok{    cov\_matrix }\OperatorTok{=}\NormalTok{ rho }\OperatorTok{**}\NormalTok{ np.}\BuiltInTok{abs}\NormalTok{(np.subtract.outer(np.arange(num\_features), np.arange(num\_features)))}

    \CommentTok{\# Perform eigen decomposition}
\NormalTok{    eigenvalues, eigenvectors }\OperatorTok{=}\NormalTok{ np.linalg.eig(cov\_matrix)}

    \CommentTok{\# Sort eigenvalues and eigenvectors in descending order}
\NormalTok{    sort\_indices }\OperatorTok{=}\NormalTok{ np.argsort(eigenvalues)[::}\OperatorTok{{-}}\DecValTok{1}\NormalTok{]}
\NormalTok{    eigenvalues }\OperatorTok{=}\NormalTok{ eigenvalues[sort\_indices]}
\NormalTok{    eigenvectors }\OperatorTok{=}\NormalTok{ eigenvectors[:, sort\_indices]}

    \CommentTok{\# Calculate V and A}
\NormalTok{    V }\OperatorTok{=}\NormalTok{ eigenvectors}
\NormalTok{    a }\OperatorTok{=}\NormalTok{ eigenvalues}


    \ControlFlowTok{return}\NormalTok{ cov\_matrix, V, a}

\KeywordTok{def}\NormalTok{ gen\_X(n, num\_covariates, intercept}\OperatorTok{=}\VariableTok{True}\NormalTok{):}
\NormalTok{    X }\OperatorTok{=}\NormalTok{ torch.randn((n, num\_covariates))}
    \ControlFlowTok{if}\NormalTok{ intercept }\KeywordTok{and}\NormalTok{ num\_covariates }\OperatorTok{\textgreater{}} \DecValTok{0}\NormalTok{:}
\NormalTok{        X[:,}\DecValTok{0}\NormalTok{] }\OperatorTok{=} \DecValTok{0}
    \ControlFlowTok{return}\NormalTok{ X}

\CommentTok{\# Setup generate data}
\CommentTok{\# {-}{-}{-}{-}{-}{-}{-}{-}{-}{-}{-}{-}{-}}
\NormalTok{n }\OperatorTok{=} \DecValTok{5000}
\NormalTok{num\_features }\OperatorTok{=} \DecValTok{50}
\NormalTok{num\_covariates }\OperatorTok{=} \DecValTok{0}
\NormalTok{num\_latents }\OperatorTok{=} \DecValTok{50}

\NormalTok{Sigma, V, A }\OperatorTok{=}\NormalTok{ gen\_Sigma(num\_features, rho}\OperatorTok{=}\FloatTok{0.8}\NormalTok{)}

\CommentTok{\# Generate the models}
\NormalTok{model\_true }\OperatorTok{=}\NormalTok{ low\_rank\_MPLN.MPLN(num\_latents, num\_features, num\_covariates, V[:,:num\_latents], np.log(A[:num\_latents]))}
\NormalTok{X }\OperatorTok{=}\NormalTok{ gen\_X(n, num\_covariates, intercept}\OperatorTok{=}\VariableTok{True}\NormalTok{)}
\NormalTok{data }\OperatorTok{=}\NormalTok{ model\_true.sample(n, X, poisson}\OperatorTok{=}\VariableTok{True}\NormalTok{)}


\CommentTok{\# Train the model}
\NormalTok{model }\OperatorTok{=}\NormalTok{ low\_rank\_MPLN.MPLN(num\_latents, num\_features, num\_covariates, V[:,:num\_latents])}
\CommentTok{\# For quicker convergence we start at the true value (this is not necessary but we don\textquotesingle{}t need to check convergence in details for each simu) }
\NormalTok{fit }\OperatorTok{=}\NormalTok{ model.train\_model(X, data, A\_init }\OperatorTok{=}\NormalTok{ model\_true.a.clone(), verbose}\OperatorTok{=}\VariableTok{True}\NormalTok{, num\_epochs}\OperatorTok{=}\DecValTok{1000}\NormalTok{)}


\CommentTok{\# Save and load}
\NormalTok{low\_rank\_MPLN.save\_model\_fit(sim\_name}\OperatorTok{=}\StringTok{"test\_sim"}\NormalTok{, sim\_iter}\OperatorTok{=}\DecValTok{123}\NormalTok{, sim\_path}\OperatorTok{=}\StringTok{""}\NormalTok{, model}\OperatorTok{=}\NormalTok{model, fit}\OperatorTok{=}\NormalTok{fit)}
\NormalTok{m, f }\OperatorTok{=}\NormalTok{ low\_rank\_MPLN.load\_model\_fit(}\StringTok{"test\_sim"}\NormalTok{, }\DecValTok{123}\NormalTok{)}
\end{Highlighting}
\end{Shaded}




\end{document}
